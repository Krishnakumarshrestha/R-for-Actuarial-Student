% Options for packages loaded elsewhere
\PassOptionsToPackage{unicode}{hyperref}
\PassOptionsToPackage{hyphens}{url}
\PassOptionsToPackage{dvipsnames,svgnames,x11names}{xcolor}
%
\documentclass[
  letterpaper,
  DIV=11,
  numbers=noendperiod]{scrreprt}

\usepackage{amsmath,amssymb}
\usepackage{lmodern}
\usepackage{iftex}
\ifPDFTeX
  \usepackage[T1]{fontenc}
  \usepackage[utf8]{inputenc}
  \usepackage{textcomp} % provide euro and other symbols
\else % if luatex or xetex
  \usepackage{unicode-math}
  \defaultfontfeatures{Scale=MatchLowercase}
  \defaultfontfeatures[\rmfamily]{Ligatures=TeX,Scale=1}
\fi
% Use upquote if available, for straight quotes in verbatim environments
\IfFileExists{upquote.sty}{\usepackage{upquote}}{}
\IfFileExists{microtype.sty}{% use microtype if available
  \usepackage[]{microtype}
  \UseMicrotypeSet[protrusion]{basicmath} % disable protrusion for tt fonts
}{}
\makeatletter
\@ifundefined{KOMAClassName}{% if non-KOMA class
  \IfFileExists{parskip.sty}{%
    \usepackage{parskip}
  }{% else
    \setlength{\parindent}{0pt}
    \setlength{\parskip}{6pt plus 2pt minus 1pt}}
}{% if KOMA class
  \KOMAoptions{parskip=half}}
\makeatother
\usepackage{xcolor}
\setlength{\emergencystretch}{3em} % prevent overfull lines
\setcounter{secnumdepth}{5}
% Make \paragraph and \subparagraph free-standing
\ifx\paragraph\undefined\else
  \let\oldparagraph\paragraph
  \renewcommand{\paragraph}[1]{\oldparagraph{#1}\mbox{}}
\fi
\ifx\subparagraph\undefined\else
  \let\oldsubparagraph\subparagraph
  \renewcommand{\subparagraph}[1]{\oldsubparagraph{#1}\mbox{}}
\fi

\usepackage{color}
\usepackage{fancyvrb}
\newcommand{\VerbBar}{|}
\newcommand{\VERB}{\Verb[commandchars=\\\{\}]}
\DefineVerbatimEnvironment{Highlighting}{Verbatim}{commandchars=\\\{\}}
% Add ',fontsize=\small' for more characters per line
\usepackage{framed}
\definecolor{shadecolor}{RGB}{241,243,245}
\newenvironment{Shaded}{\begin{snugshade}}{\end{snugshade}}
\newcommand{\AlertTok}[1]{\textcolor[rgb]{0.68,0.00,0.00}{#1}}
\newcommand{\AnnotationTok}[1]{\textcolor[rgb]{0.37,0.37,0.37}{#1}}
\newcommand{\AttributeTok}[1]{\textcolor[rgb]{0.40,0.45,0.13}{#1}}
\newcommand{\BaseNTok}[1]{\textcolor[rgb]{0.68,0.00,0.00}{#1}}
\newcommand{\BuiltInTok}[1]{\textcolor[rgb]{0.00,0.23,0.31}{#1}}
\newcommand{\CharTok}[1]{\textcolor[rgb]{0.13,0.47,0.30}{#1}}
\newcommand{\CommentTok}[1]{\textcolor[rgb]{0.37,0.37,0.37}{#1}}
\newcommand{\CommentVarTok}[1]{\textcolor[rgb]{0.37,0.37,0.37}{\textit{#1}}}
\newcommand{\ConstantTok}[1]{\textcolor[rgb]{0.56,0.35,0.01}{#1}}
\newcommand{\ControlFlowTok}[1]{\textcolor[rgb]{0.00,0.23,0.31}{#1}}
\newcommand{\DataTypeTok}[1]{\textcolor[rgb]{0.68,0.00,0.00}{#1}}
\newcommand{\DecValTok}[1]{\textcolor[rgb]{0.68,0.00,0.00}{#1}}
\newcommand{\DocumentationTok}[1]{\textcolor[rgb]{0.37,0.37,0.37}{\textit{#1}}}
\newcommand{\ErrorTok}[1]{\textcolor[rgb]{0.68,0.00,0.00}{#1}}
\newcommand{\ExtensionTok}[1]{\textcolor[rgb]{0.00,0.23,0.31}{#1}}
\newcommand{\FloatTok}[1]{\textcolor[rgb]{0.68,0.00,0.00}{#1}}
\newcommand{\FunctionTok}[1]{\textcolor[rgb]{0.28,0.35,0.67}{#1}}
\newcommand{\ImportTok}[1]{\textcolor[rgb]{0.00,0.46,0.62}{#1}}
\newcommand{\InformationTok}[1]{\textcolor[rgb]{0.37,0.37,0.37}{#1}}
\newcommand{\KeywordTok}[1]{\textcolor[rgb]{0.00,0.23,0.31}{#1}}
\newcommand{\NormalTok}[1]{\textcolor[rgb]{0.00,0.23,0.31}{#1}}
\newcommand{\OperatorTok}[1]{\textcolor[rgb]{0.37,0.37,0.37}{#1}}
\newcommand{\OtherTok}[1]{\textcolor[rgb]{0.00,0.23,0.31}{#1}}
\newcommand{\PreprocessorTok}[1]{\textcolor[rgb]{0.68,0.00,0.00}{#1}}
\newcommand{\RegionMarkerTok}[1]{\textcolor[rgb]{0.00,0.23,0.31}{#1}}
\newcommand{\SpecialCharTok}[1]{\textcolor[rgb]{0.37,0.37,0.37}{#1}}
\newcommand{\SpecialStringTok}[1]{\textcolor[rgb]{0.13,0.47,0.30}{#1}}
\newcommand{\StringTok}[1]{\textcolor[rgb]{0.13,0.47,0.30}{#1}}
\newcommand{\VariableTok}[1]{\textcolor[rgb]{0.07,0.07,0.07}{#1}}
\newcommand{\VerbatimStringTok}[1]{\textcolor[rgb]{0.13,0.47,0.30}{#1}}
\newcommand{\WarningTok}[1]{\textcolor[rgb]{0.37,0.37,0.37}{\textit{#1}}}

\providecommand{\tightlist}{%
  \setlength{\itemsep}{0pt}\setlength{\parskip}{0pt}}\usepackage{longtable,booktabs,array}
\usepackage{calc} % for calculating minipage widths
% Correct order of tables after \paragraph or \subparagraph
\usepackage{etoolbox}
\makeatletter
\patchcmd\longtable{\par}{\if@noskipsec\mbox{}\fi\par}{}{}
\makeatother
% Allow footnotes in longtable head/foot
\IfFileExists{footnotehyper.sty}{\usepackage{footnotehyper}}{\usepackage{footnote}}
\makesavenoteenv{longtable}
\usepackage{graphicx}
\makeatletter
\def\maxwidth{\ifdim\Gin@nat@width>\linewidth\linewidth\else\Gin@nat@width\fi}
\def\maxheight{\ifdim\Gin@nat@height>\textheight\textheight\else\Gin@nat@height\fi}
\makeatother
% Scale images if necessary, so that they will not overflow the page
% margins by default, and it is still possible to overwrite the defaults
% using explicit options in \includegraphics[width, height, ...]{}
\setkeys{Gin}{width=\maxwidth,height=\maxheight,keepaspectratio}
% Set default figure placement to htbp
\makeatletter
\def\fps@figure{htbp}
\makeatother
\newlength{\cslhangindent}
\setlength{\cslhangindent}{1.5em}
\newlength{\csllabelwidth}
\setlength{\csllabelwidth}{3em}
\newlength{\cslentryspacingunit} % times entry-spacing
\setlength{\cslentryspacingunit}{\parskip}
\newenvironment{CSLReferences}[2] % #1 hanging-ident, #2 entry spacing
 {% don't indent paragraphs
  \setlength{\parindent}{0pt}
  % turn on hanging indent if param 1 is 1
  \ifodd #1
  \let\oldpar\par
  \def\par{\hangindent=\cslhangindent\oldpar}
  \fi
  % set entry spacing
  \setlength{\parskip}{#2\cslentryspacingunit}
 }%
 {}
\usepackage{calc}
\newcommand{\CSLBlock}[1]{#1\hfill\break}
\newcommand{\CSLLeftMargin}[1]{\parbox[t]{\csllabelwidth}{#1}}
\newcommand{\CSLRightInline}[1]{\parbox[t]{\linewidth - \csllabelwidth}{#1}\break}
\newcommand{\CSLIndent}[1]{\hspace{\cslhangindent}#1}

\KOMAoption{captions}{tableheading}
\makeatletter
\makeatother
\makeatletter
\@ifpackageloaded{bookmark}{}{\usepackage{bookmark}}
\makeatother
\makeatletter
\@ifpackageloaded{caption}{}{\usepackage{caption}}
\AtBeginDocument{%
\ifdefined\contentsname
  \renewcommand*\contentsname{Table of contents}
\else
  \newcommand\contentsname{Table of contents}
\fi
\ifdefined\listfigurename
  \renewcommand*\listfigurename{List of Figures}
\else
  \newcommand\listfigurename{List of Figures}
\fi
\ifdefined\listtablename
  \renewcommand*\listtablename{List of Tables}
\else
  \newcommand\listtablename{List of Tables}
\fi
\ifdefined\figurename
  \renewcommand*\figurename{Figure}
\else
  \newcommand\figurename{Figure}
\fi
\ifdefined\tablename
  \renewcommand*\tablename{Table}
\else
  \newcommand\tablename{Table}
\fi
}
\@ifpackageloaded{float}{}{\usepackage{float}}
\floatstyle{ruled}
\@ifundefined{c@chapter}{\newfloat{codelisting}{h}{lop}}{\newfloat{codelisting}{h}{lop}[chapter]}
\floatname{codelisting}{Listing}
\newcommand*\listoflistings{\listof{codelisting}{List of Listings}}
\makeatother
\makeatletter
\@ifpackageloaded{caption}{}{\usepackage{caption}}
\@ifpackageloaded{subcaption}{}{\usepackage{subcaption}}
\makeatother
\makeatletter
\@ifpackageloaded{tcolorbox}{}{\usepackage[many]{tcolorbox}}
\makeatother
\makeatletter
\@ifundefined{shadecolor}{\definecolor{shadecolor}{rgb}{.97, .97, .97}}
\makeatother
\makeatletter
\makeatother
\ifLuaTeX
  \usepackage{selnolig}  % disable illegal ligatures
\fi
\IfFileExists{bookmark.sty}{\usepackage{bookmark}}{\usepackage{hyperref}}
\IfFileExists{xurl.sty}{\usepackage{xurl}}{} % add URL line breaks if available
\urlstyle{same} % disable monospaced font for URLs
\hypersetup{
  pdftitle={R book for Actuarial Student},
  pdfauthor={Jane Doe},
  colorlinks=true,
  linkcolor={blue},
  filecolor={Maroon},
  citecolor={Blue},
  urlcolor={Blue},
  pdfcreator={LaTeX via pandoc}}

\title{R book for Actuarial Student}
\author{Jane Doe}
\date{3/8/2023}

\begin{document}
\maketitle
\ifdefined\Shaded\renewenvironment{Shaded}{\begin{tcolorbox}[boxrule=0pt, enhanced, sharp corners, borderline west={3pt}{0pt}{shadecolor}, breakable, interior hidden, frame hidden]}{\end{tcolorbox}}\fi

\renewcommand*\contentsname{Table of contents}
{
\hypersetup{linkcolor=}
\setcounter{tocdepth}{2}
\tableofcontents
}
\bookmarksetup{startatroot}

\hypertarget{preface}{%
\chapter*{Preface}\label{preface}}
\addcontentsline{toc}{chapter}{Preface}

\emph{As a student studying for actuarial exams, you know that
statistical programming plays an important role in the field. R is a
widely used programming language for statistical computing and analysis,
and is a required topic in many actuarial exams. However, learning R can
be daunting, especially if you have little to no programming
experience.}

\emph{This book is designed to help you learn R programming in a way
that is relevant to your actuarial studies. It covers the essential
concepts you need to know to confidently use R in your actuarial exam.
From basic arithmetic operations to advanced data manipulation and
modeling techniques, this book provides a comprehensive guide to R
programming for actuarial students.}

\emph{Each chapter includes examples and exercises that are specifically
designed to prepare you for actuarial exams. The exercises cover topics
such as probability distributions, statistical inference, and financial
modeling, which are all important topics in the actuarial profession.}

\emph{The book assumes no prior programming experience, making it
accessible to students from all backgrounds. It is also written in a
clear and concise style, with a focus on practical applications rather
than theoretical concepts.}

\emph{We wrote this book based on our experience as actuarial students.
We believe that learning R is an essential skill for anyone studying for
actuarial exams, and We hope this book will help you achieve your
goals.}

\emph{We would like to thank our colleagues and students for their
feedback and support throughout the writing process. We would also like
to thank the actuarial community for creating such a challenging and
rewarding profession.}

\emph{Finally, we encourage readers to share their feedback and
suggestions for future editions of the book. Please feel free to contact
us at {[}email address{]} with any comments or questions.}

\emph{Best of luck with your studies!}

\emph{{[}Your Name{]}}

\emph{\hfill\break
}

\bookmarksetup{startatroot}

\hypertarget{chapter-1-introduction}{%
\chapter{Chapter 1 : Introduction}\label{chapter-1-introduction}}

R is a free, open-source programming language and software environment
for statistical computing, data analysis, and graphics. It has become
increasingly popular among data analysts, statisticians, and researchers
due to its ease of use, flexibility, and powerful capabilities.

For actuarial students, R is an essential tool for data analysis,
modeling, and simulation. Actuaries use statistical techniques and
models to evaluate and manage financial risks, and R provides a wide
range of tools for performing these tasks.

R allows actuarial students to perform statistical analysis, generate
reports, and create visualizations that can help them communicate
complex financial concepts to stakeholders. With R, students can easily
handle large data sets, perform statistical modeling, and make
predictions about future events.

Moreover, R is becoming more widely used in the actuarial industry, with
many employers now requiring their employees to have a working knowledge
of the language. As such, learning R can give actuarial students a
competitive edge in the job market and help them advance in their
careers.

In summary, R is important for actuarial students because it provides
them with a powerful tool for data analysis, modeling, and simulation
that is increasingly in demand in the actuarial profession.

\hypertarget{install-r-and-rstudio}{%
\subsection{Install R and RStudio:}\label{install-r-and-rstudio}}

Installing R and RStudio is a relatively straightforward process. Here
are the steps to follow:

\begin{itemize}
\item
  Installing R
\item
  Go to the official R project website at
  {[}**https://www.r-project.org/**{]}(https://www.r-project.org/)
\item
  Click on the ``CRAN'' link under the ``Download'' section on the
  left-hand side of the page
\item
  Select a mirror location near you
\item
  Choose the appropriate installation file for your operating system
  (e.g., Windows, Mac, or Linux)
\item
  Download and run the installation file
\item
  Follow the prompts to complete the installation
\item
  Installing RStudio
\item
  Go to the official RStudio website at
  {[}**https://www.rstudio.com/products/rstudio/download/**{]}(https://www.rstudio.com/products/rstudio/download/)
\item
  Choose the appropriate version of RStudio for your operating system
  (e.g., Windows, Mac, or Linux)
\item
  Download and run the installation file
\item
  Follow the prompts to complete the installation
\end{itemize}

Once you have both R and RStudio installed, you can launch.RStudio and
start using R to perform data analysis, modeling, and simulation. In
RStudio, you can write and run R code in the Console, which is located
in the bottom left-hand corner of the screen. You can also create and
save R scripts, plots, and reports using the various tabs in the top
right-hand corner of the screen.

\bookmarksetup{startatroot}

\hypertarget{introduction}{%
\chapter{Introduction}\label{introduction}}

This is a book created from markdown and executable code.

See Knuth (1984) for additional discussion of literate programming.

\begin{Shaded}
\begin{Highlighting}[]
\DecValTok{1} \SpecialCharTok{+} \DecValTok{1}
\end{Highlighting}
\end{Shaded}

\begin{verbatim}
[1] 2
\end{verbatim}

\bookmarksetup{startatroot}

\hypertarget{summary}{%
\chapter{Summary}\label{summary}}

In summary, this book has no content whatsoever.

\begin{Shaded}
\begin{Highlighting}[]
\DecValTok{1} \SpecialCharTok{+} \DecValTok{1}
\end{Highlighting}
\end{Shaded}

\begin{verbatim}
[1] 2
\end{verbatim}

\bookmarksetup{startatroot}

\hypertarget{references}{%
\chapter*{References}\label{references}}
\addcontentsline{toc}{chapter}{References}

\hypertarget{refs}{}
\begin{CSLReferences}{1}{0}
\leavevmode\vadjust pre{\hypertarget{ref-knuth84}{}}%
Knuth, Donald E. 1984. {``Literate Programming.''} \emph{Comput. J.} 27
(2): 97--111. \url{https://doi.org/10.1093/comjnl/27.2.97}.

\end{CSLReferences}



\end{document}
